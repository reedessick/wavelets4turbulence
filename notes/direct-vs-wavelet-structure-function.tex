\documentclass{article}

%-------------------------------------------------

\usepackage{fullpage}

\usepackage{amsfonts}
\usepackage{amssymb}
\usepackage{amsmath}

%-------------------------------------------------
\begin{document}
%-------------------------------------------------

We wish to build intuition for the relationship between the ``direct'' structure function and the ``wavelet'' structure function.

The direct structure function is defined by
\begin{equation}
    S^p_\tau(\vec{\phi}) = \left< \left| \vec{\phi}(\vec{x} + \vec{\tau}) - \vec{\phi}(\vec{x}) \right|^p \right>
\end{equation}
where $\tau = |\vec{\tau}|$, $|\cdot|$ denotes the magnitude of a vector or the absolute value of a scalar, and $\left< \cdot \right>$ denotes an ensemble average over turbulent realizations.
In practice, this ensemble average is estimated with a spatial average over $\vec{x}$.

The wavelet structure function's definition depends on the wavelet.
We use the Haar wavelet because of its simplicity (and hope that the intuition generalizes to other wavelet families).
The structure function computed via the detail coefficients of the Haar decomposition at scale $\tau$ is
\begin{equation}
    W^p_\tau(\vec{\phi}) = \left< \left| \left(\frac{1}{\tau}\sum\limits_{i=\tau}^{2\tau} \vec{\phi}(\vec{x}_i)\right) - \left(\frac{1}{\tau}\sum\limits_{i}^{\tau} \vec{\phi}(\vec{x}_i)\right)\right|^p \right>
\end{equation}
since the Haar decomposition performs a straightforward average of neighboring data when generating the approximate coefficients.

In general, expanding the sum is hard (although it is just a multinomial expansion) and may not provide any useful insight.
We focus on the 2nd order structure function ($p=2$) as it remains tractable and provides some insight into the observed differences between $S^p_\tau$ and $W^p_\tau$.
To wit, consider the scalar field $\phi_i = \phi(x_i)$
\begin{align}
    W^2_\tau
        & = \left< \frac{1}{\tau^2} \left| \sum\limits_{i}^{\tau} \left[\phi_{i+\tau} - \phi_{i}\right] \right|^2\right> \nonumber \\
        & = \tau^{-2} \left< \sum\limits_{i}^{\tau} \sum\limits_{j}^{\tau} (\phi_{i+s} - \phi_i)(\phi_{j+s} - \phi_j) \right> \nonumber \\
        & = \tau^{-2} \sum\limits_{i}^{\tau} \sum\limits_{j}^{\tau} \left< \phi_{i+\tau}\phi_{j+\tau} + \phi_{i}\phi_{j} - \phi_{i}\phi_{j+\tau} - \phi_{i+\tau}\phi_{j}\right> \nonumber \\
        & = \frac{2}{\tau^2} \sum\limits_{i}^{\tau} \sum\limits_{j}^{\tau} \left<\phi_i \phi_j\right> - \left<\phi_i \phi_{j+\tau}\right> \nonumber \\
        & = \frac{2}{\tau} \sum\limits_{j}^{\tau} \left[ \left<\phi_0 \phi_j\right> - \left<\phi_0 \phi_{j+\tau}\right> \right]
\end{align}
Recalling that $S^2_\tau = \left<(\phi_\tau - \phi_0)^2\right>$ and therefore $\left<\phi_0 \phi_\tau\right> = \left<\phi_0^2\right> - S^2_\tau/2$,
we obtain
\begin{align}
    W^2_\tau
        & = \frac{2}{\tau}\sum\limits_j^\tau \left[ \left(\left<\phi_0^2\right> - S^2_j/2\right) - \left(\left<\phi_0^2\right> - S^2_{j+\tau}/2\right)\right] \nonumber \\
        & = \frac{1}{\tau} \sum\limits_j^\tau \left[ S^2_{j+\tau} - S^2_j \right] \label{eq:result}
\end{align}
Following similar steps, it is also possible to show that the same relation holds for $S^2_\tau$ and $W^2_\tau$ for vector fields.

Eq.~\ref{eq:result} shows that we can think of $W^2_\tau$ as an average of the differences between $S^2_{\tau^\prime}$ at different scales all the way up to $\tau^\prime = 2\tau$.
Heuristically, this seems reasonable, as $W^p_\tau$ involves the difference between points as far away as $2\tau$ for the Haar decomposition.
More generally, if the wavelet has support over $m$ points, then we may expect $W^p_\tau$ to involve points that are separated by $m\tau$ and therefore depend on $S^p_{\tau^\prime}$ up to $\tau^\prime = m\tau$.

Additionally, we see that $W^2_\tau$ can be expressed as a difference between $S^2_{\tau^\prime}$ evaluated at scales separated by $\tau$.
This also explains why $W^2_\tau$ is often lower than $S^2_\tau$ for large $\tau$.
If the flucutations in the field occur at small scales, then $S^p_\tau$ will not change much as $\tau$ changes as long as $\tau$ is very large.
As such, $S^2_\tau$ will be relatively flat for large $\tau$ and therefore $W^2_\tau$ can decrease significantly.
Similarly, the wavelet decomposition will naturally average away the small-scale fluctuations in computing the approximant coefficients, which naturally explains why there is less power attributed to large-scale fluctuations.

%-------------------------------------------------
\end{document}

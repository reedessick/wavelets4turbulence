\documentclass{article}

%-------------------------------------------------

\usepackage{fullpage}

\usepackage{amsfonts}
\usepackage{amssymb}
\usepackage{amsmath}

%-------------------------------------------------
\begin{document}
%-------------------------------------------------

We wish to build intuition for the relationship between the ``direct'' structure function and the ``wavelet'' structure function.

The direct structure function is defined by
\begin{equation}
    S^p_\tau(\vec{\phi}) = \left< \left| \vec{\phi}(\vec{x} + \vec{\tau}) - \vec{\phi}(\vec{x}) \right|^p \right>
\end{equation}
where $\tau = |\vec{\tau}|$, $|\cdot|$ denotes the magnitude of a vector or the absolute value of a scalar, and $\left< \cdot \right>$ denotes an ensemble average over turbulent realizations.
In practice, this ensemble average is estimated with a spatial average over $\vec{x}$.

The wavelet structure function's definition depends on the wavelet.
We use the Haar wavelet because of its simplicity (and hope that the intuition generalizes to other wavelet families).
The structure function computed via the detail coefficients of the Haar decomposition at scale $\tau = 2^t$ is
\begin{equation}
    W^p_\tau(\vec{\phi}) = \left< \left| \left(\frac{1}{2^t}\sum\limits_{i=2^t}^{2^{t+1}} \vec{\phi}(\vec{x}_i)\right) - \left(\frac{1}{2^t}\sum\limits_{i=1}^{i=2^t} \vec{\phi}(\vec{x}_i)\right)\right|^p \right>
\end{equation}
since the Haar decomposition performs a straightforward average of neighboring data when generating the approximate coefficients.

In general, expanding the sum is hard (although it is just a multinomial expansion) and may not provide any useful insight.
We focus on the 2nd order structure function ($p=2$) as it remains tractable and provides some insight into the observed differences between $S^p_\tau$ and $W^p_\tau$.
To wit,
\begin{align}
    W^2_\tau
        & = \left< \frac{1}{\tau^2} \left| \sum\limits_{i}^{\tau} \left[\phi_{i+\tau} - \phi_{i}\right] \right|^2\right> \\
        & = \tau^{-2} \left< \sum\limits_{i}^{\tau} \sum\limits_{j}^{\tau} (\phi_{i+s} - \phi_i)(\phi_{j+s} - \phi_j) \right> \\
        & = \tau^{-2} \sum\limits_{i}^{\tau} \sum\limits_{j}^{\tau} \left< \phi_{i+\tau}\phi_{j+\tau} + \phi_{i}\phi_{j} - \phi_{i}\phi_{j+\tau} - \phi_{i+\tau}\phi_{j}\right> \\
        & = \frac{2}{\tau^2} \sum\limits_{i}^{\tau} \sum\limits_{j}^{\tau} \left<\phi_i \phi_j\right> - \left<\phi_i \phi_{j+\tau}\right> \\
        & = \frac{2}{\tau} \sum\limits_{j}^{\tau} \left[ \left<\phi_0 \phi_j\right> - \left<\phi_0 \phi_{j+\tau}\right> \right] \\
\end{align}
Recalling that $S^2_\tau = \left<(\phi_\tau - \phi_0)^2\right>$ and therefore $\left<\phi_0 \phi_\tau\right> = \left<\phi_0^2\right> - S^2_\tau/2$,
we obtain
\begin{align}
    W^2_\tau
        & = \frac{2}{\tau}\sum\limits_j^\tau \left[ \left(\left<\phi_0^2\right> - S^2_j/2\right) - \left(\left<\phi_0^2\right> - S^2_{j+\tau}/2\right)\right] \\
        & = \frac{1}{\tau} \sum\limits_j^\tau \left[ S^2_{j+\tau} - S^2_j \right]
\end{align}

\textbf{
I believe this for scalar fields, but what about vector fields?
}

\textbf{
Discuss the intuition: $W^p_\tau$ depends on $S^p_{\tau^\prime}$ for all $\tau^\prime \leq 2\tau$.
In general, for wavelets with length $m$, we expect $W^p_\tau$ to depend on $S^p_{\tau^\prime}$ for all $\tau^\prime \leq m \tau$.
}

%-------------------------------------------------
\end{document}

\documentclass{article}

%-------------------------------------------------

\usepackage{amsfonts}
\usepackage{amssymb}
\usepackage{amsmath}

%-------------------------------------------------
\begin{document}
%-------------------------------------------------

We wish to build intuition for the relationship between the ``direct'' structure function and the ``wavelet'' structure function.

The direct structure function is defined by
\begin{equation}
    S^p_\tau(\vec{\phi}) = \left< \left| \vec{\phi}(\vec{x} + \vec{\tau}) - \vec{\phi}(\vec{x}) \right|^p \right>
\end{equation}
where $\tau = |\vec{\tau}|$ and $\left< \cdot \right>$ denotes an ensemble average over turbulent realizations.
In practice, this ensemble average is estimated with a spatial average over $\vec{x}$.

The wavelet structure function's definition depends on the wavelet.
We use the Haar wavelet because of its simplicity (and hope that the intuition generalizes to other wavelet families).
The structure function computed via the detail coefficients of the Haar decomposition at scale $\tau$ is
\begin{equation}
    something
\end{equation}

%-------------------------------------------------
\end{document}
